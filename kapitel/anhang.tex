\chapter*{Anhang}
\thispagestyle{plain}
\addcontentsline{toc}{chapter}{Anhang}

\section{Aufbau des Generators}
\begin{python}
Generator(
(block1): Sequential(
(0): Conv3d(2, 4, kernel_size=(5, 5, 5), stride=(1, 1, 1), padding=(2, 2, 2))
(1): BatchNorm3d(4, eps=1e-05, momentum=0.1, affine=True, track_running_stats=True)
(2): ReLU(inplace=True)
(3): Conv3d(4, 4, kernel_size=(5, 5, 5), stride=(1, 1, 1), padding=(2, 2, 2))
(4): BatchNorm3d(4, eps=1e-05, momentum=0.1, affine=True, track_running_stats=True)
)
(shortcut1): Sequential(
(0): Conv3d(2, 4, kernel_size=(1, 1, 1), stride=(1, 1, 1))
(1): BatchNorm3d(4, eps=1e-05, momentum=0.1, affine=True, track_running_stats=True)
)
(block2): Sequential(
(0): Conv3d(4, 8, kernel_size=(5, 5, 5), stride=(1, 1, 1), padding=(2, 2, 2))
(1): BatchNorm3d(8, eps=1e-05, momentum=0.1, affine=True, track_running_stats=True)
(2): ReLU(inplace=True)
(3): Conv3d(8, 2, kernel_size=(5, 5, 5), stride=(1, 1, 1), padding=(2, 2, 2))
(4): BatchNorm3d(2, eps=1e-05, momentum=0.1, affine=True, track_running_stats=True)
)
(shortcut2): Sequential(
(0): Conv3d(4, 2, kernel_size=(1, 1, 1), stride=(1, 1, 1))
(1): BatchNorm3d(2, eps=1e-05, momentum=0.1, affine=True, track_running_stats=True)
)
(r1): ReLU(inplace=True)
(r2): ReLU(inplace=True)
(u1): Upsample(size=(32, 32, 32), mode=nearest)
(u2): Upsample(size=(64, 64, 64), mode=nearest)
)
\end{python}
\newpage
\thispagestyle{plain}
\section{Aufbau des Discriminators}
\begin{python}
Discriminator(
(main): Sequential(
(0): Conv3d(6, 8, kernel_size=(4, 4, 4), stride=(1, 1, 1), bias=False)
(1): BatchNorm3d(8, eps=1e-05, momentum=0.1, affine=True, track_running_stats=True)
(2): LeakyReLU(negative_slope=0.01, inplace=True)
(3): Conv3d(8, 16, kernel_size=(4, 4, 4), stride=(1, 1, 1), bias=False)
(4): BatchNorm3d(16, eps=1e-05, momentum=0.1, affine=True, track_running_stats=True)
(5): LeakyReLU(negative_slope=0.01, inplace=True)
(6): Conv3d(16, 1, kernel_size=(4, 4, 4), stride=(1, 1, 1), padding=(2, 2, 2), bias=False)
(7): Flatten(start_dim=1, end_dim=-1)
(8): LazyLinear(in_features=0, out_features=1, bias=True)
(9): Sigmoid()
)
)
\end{python}
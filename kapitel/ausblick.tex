\chapter{Ausblick}
\thispagestyle{fancy}

In dieser Arbeit wurde aufzeigt, dass eine generelle Implementierung von neuronalen Netzen im Bereich der 3D-Simulationen für VFX möglich ist. Eine große Herausforderung war jedoch die Verarbeitung der Daten und das stabile Trainieren eines KI-Modells. Hier ergab sich Potenzial für zukünftige Studien, die sich der Vereinfachung des OpenVDB-NumPy-PyTorch Workflows widmen oder an einer Lösung für eine variierende Input-Anzahl in neuronalen Netzen forschen. Da das OpenVDB Dateiformat nur aktive Voxel speichert und sich die Auflösung jeder Dimension pro Frame ändern kann, würde eine variierende Input-Anzahl eine Simplifizierung in der Datenverarbeitung darstellen.\\

Ebenfalls blieb die Frage offen, ob ein neuronales Netz anhand des Density und Temperature Fields ein visuell artefaktfreies Ergebnis beim Upsampling erzielen könnte und ob es in der Lage wäre, eine physikalische Verbindung zwischen diesen beiden Fields zu erkennen. Dem schließt sich die Frage an, ob ein neuronales Netz, das auf einem Datensatz von Simulationen einer Explosion zufriedenstellend trainiert wurde, ebenso gute Ergebnisse erzeugen könnte, wie wenn es dazu genutzt werden würde, die Rauchsimulation einer Kerze zu skalieren. \\
Weitere Forschungsbereiche ergeben sich innerhalb der verschiedenen Programme zur Erzeugung von Rauchsimulationen. Anstatt die Berechnungen für den nächsten Frame einer Simulation auf den Voxeln zu basieren, ließen sich eventuell die Parameter in ein neuronales Netz schleusen, die für die Berechnung der Voxel zuständig sind. Möglicherweise ist ein neuronales Netz so in der Lage Werte für Formeln und Zeitintervalle von physikalischen Strömungen als Input zu erhalten und erzeugt darauf basierend einen Frame bestehend aus Voxeln. \\

Allgemein bleiben Rauchsimulationen ein Forschungsgebiet, das einige Forschungspotentiale für Effizienzsteigerung bietet. Die Möglichkeit Simulationen schneller durchzuführen oder sie auf andere Weise mit neuronalen Netzen zu manipulieren, kann finanzielle Vorteile für ein Unternehmen oder mehr kreative Freiheit für einen 3D-Artist bedeuten. Mit steigender Leistungsfähigkeit von Computerhardware kann es in Zukunft möglich sein, die Dauer solcher Forschungen zu verkürzen und sie allgemein zugänglicher zu machen. Somit entstehen eine Vielzahl neuer Anwendungen, die Arbeitsabläufe beschleunigen und die Arbeit eines Artists vereinfachen können.
\chapter{Einführung}
\thispagestyle{fancy}

Von Jahr zu Jahr macht sich ein rasanter Anstieg der Nutzung von künstlicher Intelligenz im Alltag bemerkbar \parencite[]{fraunhofer-allianz-big-data-2017}. Sie erstellt Jahresrückblicke basierend auf den Fotos in einer Smartphone-Galerie \parencite[]{ralf-2022} oder ist in der Lage ganze Bücher wie \textit{Alice im Wunderland} innerhalb einer Seite zusammenzufassen \parencite[]{openai-2022}.
In Hollywood haben diese Technologien ebenfalls ihren Einzug gefunden. Aufgaben, die einen hohen zeitlichen Aufwand von menschlichen Arbeitern erfordern, können mithilfe von künstlicher Intelligenz nun in wenigen Minuten durchgeführt werden. Einer dieser Arbeitsschritte ist zum Beispiel das \textit{Rotoscoping}, das eingesetzt wird, um Darsteller aus dem Filmbild auszuschneiden, damit zwischen ihnen und dem Hintergrund neue Elemente einfügen werden können \parencite[]{dawson-2021}. In der Vergangenheit bedeutete dieses Ausschneiden von Elementen aus einem Filmbild, das jedes Bild einzeln bearbeitet werden musste. Mithilfe von künstlicher Intelligenz hingegen lassen sich hunderte Filmbilder in kurzer Zeit komplett automatisiert ausschneiden \parencite[]{befores-afters-2021}.   \\

Im Bereich der Visual Effects existieren ebenso eine Vielzahl an zeitintensiven Tätigkeiten, insbesondere das Simulieren von Wasser oder großen Explosionen. Die Berechnung einer Iteration der von den Alps Studios erstellten Explosion dauerte fünf Stunden \parencite[]{alps-vfx-2020}. Simulationszeiten in dieser Größe sind keine Seltenheit und es müssen oft mehrere Varianten einer Simulation erstellt, oder Verbesserungen umgesetzt werden, was ein erneutes Simulieren mit sich zieht. Deshalb besteht ein allgemeiner Bedarf an zeitsparenden Optimierungen.\\

Diese Forschungsarbeit widmet sich der Frage, ob sich die zeitintensiven Berechnungen von Simulationen im Visual Effects Bereich mit dem Einsatz von künstlicher Intelligenz verkürzen lassen. Dabei soll die Methode des sogenannten \textit{Upsamplings} untersucht werden. Laut \citet[]{chaubey-2021} würden beim Prozess des Upsamplings Daten erweitert werden. Beispielsweise wird die Auflösung eines Pixelbildes erhöht, indem künstlich erzeugte Pixel zwischen den bereits vorhandenen Pixeln eingesetzt werden. Der Ausdruck des \textit{Hochskalierens} soll in den nächsten Abschnitten ebenfalls für das Upsampling stehen.\\

Es soll untersucht werden, ob sich dieses Verfahren im Zusammenspiel mit künstlicher Intelligenz auch für 3D-Simulationen einsetzen lässt. Somit wäre es im besten Fall möglich, eine Explosion in geringer Auflösung, die einen geringen Detailgrad aufweist, mithilfe von künstlicher Intelligenz auf eine höhere Auflösung zu skalieren, um so den Detailgrad zu verbessern. Dieser Prozess soll keine visuellen Artefakte erzeugen, einen Zeitvorteil gegenüber dem Simulieren haben und ebenfalls soll die künstliche Intelligenz mit geringen Dateigrößen arbeiten können. Somit ergibt sich die Forschungsfrage, ob ein Upsampling von Rauchsimulationen mittels künstlicher Intelligenz unter den Aspekten der visuellen Artefaktfreiheit, Zeiteffizienz und Datensparsamkeit gegenüber dem direkten Simulieren in hoher Auflösung möglich ist?\\

Diese Forschung bietet eine hohe Relevanz innerhalb des Visual Effects Bereichs, da ein schnelleres Berechnen und Iterieren von Simulationen finanzielle Vorteile und mehr kreative Freiheiten bietet. So macht es einen deutlichen Unterschied, ob ein Rechner in einem Visual Effects Unternehmen für mehrere Stunden eine Simulation berechnet und somit nicht für andere Aufgaben zu Verfügung steht oder ob dieser mehrere Simulationen in derselben Zeit mittels Upsampling bearbeiten kann. Auch die Arbeiter, die solche Simulationen erstellen, profitieren von kürzeren Simulationszeiten, da sie somit mehrere Versionen und Änderungswünsche bearbeiten können.\\

Diese Forschungsarbeit entsteht im Rahmen des Bachelorstudiengangs \textit{Audiovisuelle Medien} an der \textit{Hochschule der Medien} und dient der Erlangung des Grades \textit{Bachelor of Engineering}.
Das Ziel der Arbeit soll sein, die Forschungsfrage anhand verschiedener Gütekriterien zu evaluieren und dem Rezipient einen Überblick über die Themen Visual Effects, 3D-Simulationen und künstliche Intelligenz zu geben. \\

Die Forschungsmethode dieser Arbeit ist ein Experiment, für das ein eigener Datensatz an 3D-Simulationen erstellt wird, welcher als Grundlage für das Training eines neuronalen Netzes dient.\\

Der Aufbau dieser Forschungsarbeit beginnt mit einer Einführung in den Bereich der Visual Effects und der \textit{Computergenerierten Bilder}. Darauf aufbauend folgt ein Abschnitt über die Erstellung von 3D-Simulationen, deren Arten und dem zugrundeliegenden Dateiformat, das zur Speicherung der Simulationen eingesetzt wird. Ein weiterer Abschnitt handelt von künstlicher Intelligenz und den Aufbau von neuronalen Netzen. Im Hauptteil dieser Arbeit wird die Forschungsmethodik und die Implementierung der künstlichen Intelligenz erläutert. Anschließend werden die Ergebnisse diskutiert und in einem Fazit zusammengefasst. Im letzten Abschnitt wird ein Ausblick auf weiterführende Forschungen gegeben, die in diesem Bereich möglich sind.  


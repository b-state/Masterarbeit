\thispagestyle{fancy}
\chapter{Vorwort}
\chapter{Einleitung}
\section{Hintergrund und Motivation}
\subsection{Bedeutung des Design Thinking-Prozesses für Startups}
\subsection{Herausforderungen bei der Selbstevaluation des Startup-Fortschritts}
\subsection{Relevanz eines strukturierten Reifegradmodells}

\section{Problemstellung und Zielsetzung}
\subsection{Fehlende standardisierte Methode zur Startup-Selbsteinschätzung}
\subsection{Ziel: Entwicklung eines adaptiven Fragebogens zur Evaluation des Design Thinking-Prozesses eines Startups}

\section{Forschungsfragen}
\subsection{Wie kann ein dynamischer Fragebogen entwickelt werden, der sich an den Antworten der Nutzer orientiert?}
\subsection{Welche Metriken sind geeignet, um den Fortschritt eines Startups im Design Thinking-Prozess zu messen?}

\section{Aufbau der Arbeit}
\subsection{Überblick über die Kapitelstruktur}

\chapter{Theoretische Grundlagen}

\section{Grundlagen des Design Thinking-Prozesses}
\subsection{Definition und Phasen des Design Thinking}
\subsection{Unterschiede zu anderen Innovationsmethoden}
\subsubsection{Lean Startup}
\subsubsection{Weitere Methoden}

\section{Reifegradmodelle in der Startup-Forschung}
\subsection{Bestehende Reifegradmodelle}
\subsubsection{Schwächen dieser Modelle}
\subsection{Anforderungen an ein praxistaugliches Reifegradmodell für Startups}

\section{Fragebögen zur Selbsteinschätzung}
\subsection{Grundlagen zu Fragebögen mit Selbstkontrolle}
\subsubsection{Kombination aus qualitativen und quantitativen Fragen}
\subsection{Adaptive Fragebogen-Systeme: Dynamische Anpassung der Fragen}
\subsubsection{Gezielte Rückfragen}
\subsection{Definition von Metriken zur Startup-Evaluation}

\chapter{Stand der Forschung}

\chapter{Methodik}

\section{Experteninterviews zur Identifikation von Selbstüberschätzung}
\section{Entwicklung eines Fragebogens mit und ohne adaptive Nachfragen}
\section{Evaluation durch  Befragungsrunden mit Startups}

\chapter{Entwicklung des Fragebogens}

\section{Anforderungen an den Fragebogen}
\subsection{Strukturierte Erfassung des Design Thinking-Fortschritts}
\subsection{Kriterien für die Formulierung von Fragen und Antwortoptionen}

\section{Dynamische Anpassung der Fragen}
\subsection{Konzept eines adaptiven Fragebogens}
\subsection{Definition von Regeln für Folgefragen und Feedback-Mechanismen}

\section{Technische Umsetzung als Web-Tool}
\subsection{Architektur des webbasierten Fragebogens}
\subsection{Technologien und Frameworks für die Entwicklung}

\chapter{Evaluation und Analyse}

\section{Testphase mit Startups}
\subsection{Durchführung des Fragebogens mit Startup-Teams}
\subsection{Datenerhebung}

\section{Auswertung der Ergebnisse}
\subsection{Analyse der erfassten Daten}

\section{Vergleich mit bestehenden Methoden}
\subsection{Diskussion der Vorteile des entwickelten Modells im Vergleich zu bestehenden Ansätzen}

\chapter{Fazit und Ausblick}

\section{Zusammenfassung der Erkenntnisse}
\subsection{Bedeutung adaptiver Fragebögen für Startup-Selbsteinschätzungen}
\subsection{Potenzial von Reifegradmodellen im Design Thinking-Kontext}

\section{Limitationen der Arbeit}
\subsection{Herausforderungen bei der Messung subjektiver Einschätzungen}
\subsection{Mögliche Verzerrungen durch Selbstauskunft}

\section{Zukünftige Forschungsansätze}
\subsection{Weiterentwicklung des Fragebogens durch Machine Learning}
\subsection{Anwendung des Modells auf andere Innovationsmethoden}
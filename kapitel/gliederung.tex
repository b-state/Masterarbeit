\thispagestyle{fancy}
\chapter{Vorwort}
\chapter{Überblick}

\section{Evaluation von Startups}
\section{Herausforderungen und Chancen}
\section{Fehlende standardisierte Methode zur Reifegradmessung von Startups}

\chapter{Ziele}
\section{Entwicklung eines adaptiven Fragebogens zur Evaluation des Reifegrades eines Startups}
\section{Forschungsfragen}


\chapter{Theoretische Grundlagen}

\section{Reifegradmodelle in der Startup-Forschung}
\subsection{Definiton Reifegradmodell}
\subsection{Relevanz eines strukturierten Reifegradmodells}
\subsection{Bedeutung der Reifegradmessung für Startups}
\subsection{Bestehende Reifegradmodelle}
\subsubsection{CMMI Ecosystem}
\subsubsection{SPICE}
\subsubsection{Customer Development}
\subsubsection{Business Model Canvas}
\subsubsection{Weitere Modelle}
\subsection{Stärken und Schwächen bestehender Modelle}
\subsection{Anforderungen an ein praxistaugliches Reifegradmodell für Startups}

\section{Der „Startup Path“ – Ein neues Reifegradmodell für Startups}
\subsection{Definition und Struktur des Modells}
\subsubsection{Ideen- und Team-Findung (Vision-Opportunity-Check)}
\subsubsection{Chancen-Validierung (Viability-Analysis)}
\subsubsection{Marktgerechte Lösungsfindung (Solution-Market-Fit)}
\subsubsection{Übergang zum Markt-Start (Transition to Launch)}
\subsubsection{Markt-Eintritt und Wachstum (Launch and Scale-Up)}
\subsection{Unterschiede zu bestehenden Reifegradmodellen}
\subsection{Praktische Relevanz und Anwendungsmöglichkeiten}


\section{Fragebögen zur Selbsteinschätzung}
\subsection{Grundlagen zu Fragebögen mit Selbstkontrolle}
\subsubsection{Kombination aus qualitativen und quantitativen Fragen}
\subsection{Adaptive Fragebogen-Systeme: Dynamische Anpassung der Fragen}
\subsubsection{Gezielte Rückfragen}
\subsection{Definition von Metriken zur Startup-Evaluation}

\chapter{Stand der Forschung}

\chapter{Methodik}

\section{Experteninterviews zur Identifikation von Selbstüberschätzung}
\section{Entwicklung eines Fragebogens mit und ohne adaptive Nachfragen}
\section{Evaluation durch  Befragungsrunden mit Startups}

\chapter{Entwicklung des Fragebogens}

\section{Anforderungen an den Fragebogen}
\subsection{Strukturierte Erfassung des Startup Paths}
\subsection{Kriterien für die Formulierung von Fragen und Antwortoptionen}

\section{Dynamische Anpassung der Fragen}
\subsection{Konzept eines adaptiven Fragebogens}
\subsection{Definition von Regeln für Folgefragen und Feedback-Mechanismen}

\section{Technische Umsetzung als Web-Tool}
\subsection{Architektur des webbasierten Fragebogens}
\subsection{Technologien und Frameworks für die Entwicklung}

\chapter{Evaluation und Analyse}

\section{Testphase mit Startups}
\subsection{Durchführung des Fragebogens mit Startup-Teams}
\subsection{Datenerhebung}

\section{Auswertung der Ergebnisse}
\subsection{Analyse der erfassten Daten}

\section{Vergleich mit bestehenden Methoden}
\subsection{Diskussion der Vorteile des entwickelten Modells im Vergleich zu bestehenden Ansätzen}

\chapter{Fazit und Ausblick}

\section{Zusammenfassung der Erkenntnisse}
\subsection{Bedeutung adaptiver Fragebögen für \\ Startup-Selbsteinschätzungen}
\subsection{Potenzial von Reifegradmodellen im Startup Kontext}

\section{Limitationen der Arbeit}
\subsection{Herausforderungen bei der Messung subjektiver Einschätzungen}
\subsection{Mögliche Verzerrungen durch Selbstauskunft}

\section{Zukünftige Forschungsansätze}
\subsection{Weiterentwicklung des Fragebogens durch Machine Learning}
\subsection{Anwendung des Modells auf andere Innovationsmethoden}
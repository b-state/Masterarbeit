\chapter*{Glossar}
\thispagestyle{plain}
\addcontentsline{toc}{chapter}{Glossar}

\begin{table}[ht]
    \begin{tabularx}{\textwidth}{lX}
3D-Sub-Array	&	Teilstück eines größeren Arrays.	\\
ADAM & Ein Optimizer Algorithmus in neuronalen Netzen. \\
Activation Function	&	Funktion die den Wert am Ausgang eines Neurons bestimmt.	\\
Agent	&	Ein Operator innerhalb eines Programmes mit der Fähigkeit zu handeln.	\\
Backpropagation	&	Fehlerrückführung in neuronalen Netzen.	\\
Batch	&	Datenpacket das durch ein neuronales Netz geschleust wird.	\\
Batch Normalisation Layer	&	Berechnet die Werte eines Batches neu und normalisiert sie.	\\
Batch Size	&	Anzahl der Daten in einem Batch.	\\
Bias	&	Verzerrung oder die Einseitigkeit eines Datensatzes.	\\
Bounding Box	&	Maximale Containergröße einer Simulation.	\\
Channels	&	Kanäle in einem Convolutional Layer.	\\
Convolutional Layer	&	Ebene in einem Convolutional Neural Network.	\\
Convolutional Neural Networks	&	Eine From neuronaler Netze für die Bildverarbeitung.	\\
Deep Learning	&	Teilgebiet des Machine Learnings, arbeitet mit neuronalen Netzen.	\\
Diskriminator	&	Teil eines Generative Adversarial Networks. Dieser klassifiziert Daten.	\\
Dropout Layer	&	Ebene in einem neuronalen Netz die Verbindungen löscht.	\\
Feature Engineering	&	Extrahieren von Merkmalen aus einem Datensatz.	\\




     \end{tabularx}
    \label{tab:my_label1}
\end{table}
\thispagestyle{plain}
\begin{table}[ht]
    \begin{tabularx}{\textwidth}{lX}
        Feature Map	&	Erzeugnis aus Convolutional Neural Networks.	\\
    Full CG & Ein Film der ausschließlich mit am Computer generierten Bildern erzeugt wurde.\\
Fully-connected-Layer	&	Ebenen in neuronalen Netzen, die mit allen anderen Neuronen der vorherigen Ebene verbunden sind.	\\
    Generative Adversarial Networks	&	Eine Art von neuronalen Netzen. Wird verwendet um Daten zu generieren.	\\
Generator	&	Teil eines Generative Adversarial Networks. Dieser erzeugt Daten.\\
    Houdini	&	3D-Software 	\\
Kernel	&	Filter in einem Convolutional Neural Network.	\\
Learning Rate	&	Maß mit dem die Gewichte eines neuronalen Netzes verfeinert werden. \\
Loss	&	Fehlerwert eines neuronalen Netzes.	\\
LowRes-3D-Array	&	In der Auflösung reduzierte Version eines 3D-Sub-Arrays.	\\
Machine Learning	&	Teilgebiet der künstlichen Intelligenz.	\\
Mantaflow	&	3D-Simulations-Software	\\
Mesh & Das Netz oder Skelett aus dem 3D-Modelle bestehen.\\
Mode Collapse	&	Eine Fehlerquelle eines Generative Adversarial Networks.	\\
Nearest Neighbor Interpolation	&	Mathematische Funktion die Werte basierend auf den Werten der Nachbarwerte erzeugt.	\\
NumPy	&	Python-Modul für Matrizenkalkulation.	\\
Optimizer	&	Algorithmus der ein neuronales Netz verbessert.	\\
Overfitting	&	Problem eines neuronalen Netzes, wenn es nur gut auf Trainingsdaten funktioniert.	\\
Pandas	&	Pyhton-Modul für Statistiken.	\\
Postproduction & Bearbeitung eines Films die nach dem Dreh stattfindet. \\


    \end{tabularx}
    \label{tab:my_label2}
\end{table}

\begin{table}[ht]
\vspace{-6cm}%
    \begin{tabularx}{\textwidth}{lX}
    PyTorch	&	Python Framework für künstliche Intelligenz.	\\
    Quads & Bauteil eines 3D-Modells.\\
    Real Film & Ein Film oder Filmteil, der nicht am Computer entstanden ist. \\
Residual Blocks	&	Ein Teil eines Residual Neural Networks.	\\
Rendern & Erzeugung eines Bildes aus 3D-Daten.\\
    Residual Neural Networks	&	Eine Art von neuronalen Netzen. Wird verwendet, um die Performance dieser zu verbessern.	\\
Rotoscoping	&	Der Prozess des Ausschneidens von Elementen aus einem Bild.	\\
Shortcut	&	Abkürzung in Residual Neural Networks.	\\
Shot & Eine Bildsequenz im Film, getrennt durch Filmschnitte. \\
    Tensorboard	&	Software für die Visualisierung von Trainingsdaten für neuronale Netze.	\\
Transposed Convolution	&	Eine Art der Convolution bei der neue Werte geschaffen werden.	\\
Triangle & Bauteil eines 3D-Modells.\\
Underfitting	&	Problem eines neuronalen Netzes, wenn es nicht gut auf Trainingsdaten funktioniert	\\
Upsampling	&	Das hochskalieren der Auflösung eines Bildes oder vergleichbarer Medien.	\\
Vanishing Gradients	&	Problem eines neuronalen Netzes, wenn die Layer am Anfang keinen Lernimpuls mehr erhalten.	\\
Voxel & Kleinstes Element einer 3D-Rauchsimulation.\\
Wireframe & Das Netz oder Skelett, aus dem 3D-Modelle bestehen.\\

    \end{tabularx}
    \label{tab:my_label3}
\end{table}




\chapter*{Kurzfassung}
\thispagestyle{plain}
\addcontentsline{toc}{chapter}{Kurzfassung}

Das Ziel der vorliegenden Arbeit ist die Beantwortung, ob es durch den Einsatz von künstlicher Intelligenz möglich sei, die Auflösung einer 3D Rauchsimulation zu erhöhen. Dazu wird folgende Forschungsfrage gestellt: Ist ein Upsampling von Rauchsimulationen mittels künstlicher Intelligenz unter den Aspekten der visuellen Artefaktfreiheit, Zeiteffizienz und Datensparsamkeit gegenüber
dem direkten Simulieren in hoher Auflösung möglich?\\

Um diese Frage zu beantworten, wurde ein Experiment durchgeführt, das mit der Erzeugung eines Datensatzes an 3D-Simulationsdaten begann. Anschließend wurde ein neuronales Netz aufgebaut und mit den Simulationsdaten trainiert. Zur Evaluation der Ergebnisse wurden die Simulationsdaten in ihrer Auflösung reduziert, danach durch die künstliche Intelligenz hochskaliert und in die ursprüngliche Auflösung konvertiert.\\

Das Ergebnis des Experimentes zeigt, dass es nicht möglich war eine künstliche Intelligenz so zu trainieren, dass diese ein Upsampling ohne visuelle Artefakte durchführen konnte. Dass ein Upsampling mit weiteren Optimierungen der neuronalen Netze aber möglich ist, konnte durch den Vergleich mit anderen Implementationen gezeigt werden.\\
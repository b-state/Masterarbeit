\chapter{Stand der Forschung}
\thispagestyle{fancy}

Die Thematik der Deep Neural Networks im Simulationsbereich ist bereits weitläufig bekannt. Einige Forscher betreiben bereits Forschung in diesem Bereich und haben entsprechende Arbeiten publiziert. Da sich durchaus Parallelen zu der vorliegenden Arbeit auftun, sollen diese im Folgenden aufgezeigt werden. Vorgestellt werden drei wissenschaftliche Publikationen, welche Ergebnisse beim Upsamling von Rauchsimulationen mittels künstlicher Intelligenz erzielen konnten. Keine dieser Arbeiten nutzten die gleichen Methoden, die in dieser Arbeit verwendet werden. Auch wurde keines der Ergebnisse in einem VFX Umfeld bezüglich der visuellen Qualität und des Dateiformates OpenVDB getestet.\\

\citet[]{baidirectory} zeigen in \textit{Dynamic Upsampling of Smoke through Dictionary based Learning}, wie die Auflösung und Details von Rauchsimulationen mittels einer Bibliothek aus vortrainierten Mustern oder sogenannten \textit{Patches} verbessert werden kann. Das neuronale Netz lernt, wie es aus einer hochaufgelösten Rauchsimulation Patches extrahieren kann, die zur selben, niedrig aufgelösten Simulation passen. Es werden dafür paarweise die zusammengehörenden Patches niedriger und hoher Auflösung gespeichert. Wird dem Netz anschließend eine niedrig aufgelöste Simulation vorgelegt, findet es passende Patches mit hoher Auflösung aus der Bibliothek und erzeugt eine hochauflösende Ausgabe. Der Vorteil dieses KI-Modells ist, dass die Ausgabe der hochskalierten Daten einen hohen Detailreichtum innerhalb der Rauchsimulation aufweist. Ebenfalls wird nur ein Density Field benötigt, um eine Ausgabe zu erzeugen. Der Nachteil dieses KI-Modells ist hingegen, dass visuelle Artefakte innerhalb der Ausgabe auftreten können. Sollte das Netz Eingaben skalieren müssen, zu denen keine passenden Patches in der zuvor trainierten Bibliothek existieren, werden Artefakte in diesen Bereichen sichtbar.\\

Laut \citet[]{Gao_2021} sei es möglich, ein solches Upsampling mittels neuronaler Netze auch ohne hochauflösende Trainingsdaten zu schaffen. Dies zeigen die Autoren in \textit{Super-resolution and denoising of fluid flow using physics-informed convolutional neural networks without high-resolution labels}. Während des Trainings bestehen die Inputs nur aus niedrig aufgelösten Datensätzen. Um das Netz zu verfeinern, wird analysiert, wie ähnlich der Output des Netzes an einer physikalisch korrekten Berechnung der Strömung ist. Die Autoren nannten dies \textit{Physics-informed training}. Der Vorteil dieses Netzes besteht darin, dass es nicht mit zwei Datensätzen, bestehend aus niedrig und hochaufgelösten Daten, trainiert werden muss und trotzdem in der Lage ist, eine hochauflösende Ausgabe zu erzeugen. Der Nachteil dieses KI-Modells sind die fehlenden Erkenntnisse im dreidimensionalen Raum, da das Netz nur mit zweidimensionalen Daten trainiert wurde. \\

Die Autoren der Publikation \textit{tempoGAN: A Temporally Coherent, Volumetric GAN for Super-resolution Fluid Flow \parencite{xie2018tempoGAN}}, trainierten ein GAN, welches es ermöglicht, ein Upsampling auf Basis des Density und Velocity Fields einer Rauchsimulaition zu erzeugen. Das Modell lernt mit niedrig aufgelösten Daten, die physikalische Beziehung zwischen Density und Velocity zu verstehen. Der Generator erzeugt, basierend auf einem niedrig aufgelösten Input, einen hochaufgelösten Output. Dabei gelang den Autoren, eine visuell artefaktfreie Qualität über mehrere Frames hinweg zu garantieren. Dies erreichten sie, indem der Generator mit drei aufeinanderfolgenden Frames trainiert wurde und somit eine temporale Kohärenz lernte. Dies stellt einen großen Vorteil des KI-Modells dar und ist im VFX-Bereich besonders wichtig. Der Nachteil dieses KI-Modells ist jedoch, dass es nicht in der Lage ist, mit Simulationsdaten aus Houdini oder anderen Programmen der VFX-Industrie zu arbeiten. Ebenfalls bedarf die Speicherung des Velocity Fields einer Simulation im Vergleich zum Density Field mehr Speicherplatz. Da der Betrieb von Servern zum Speichern der Daten einen hohen Kostenaufwand mit sich bringt, bietet sich hier die Grundlage zur Forschung, ob das Velocity Field mit dem Temperature Field einer Simulation für das Training des KI-Modells ausgetauscht werden kann. \\

Zusammenfassend lässt sich feststellen, dass keines der genannten KI-Modelle in einem VFX-Umfeld genutzt werden könnte, da sie nicht mit dreidimensionalen Daten oder dem Dateiformat OpenVDB arbeiten. Dies zeigt, dass hier ein deutlicher Bedarf an Forschung besteht und die vorliegende Arbeit das Potenzial hat, diese Lücke zu schließen.

